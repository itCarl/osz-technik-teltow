\documentclass[12pt, a4paper]{article}
\usepackage[a4paper, left=2.5cm, right=2cm, top=2cm, bottom=3cm, heightrounded, includefoot]{geometry}

% = = = = = = = = = = = = = = = = = = = = = = = = = = = = = = = = = = = = = = = = =
% |		PACKAGES
% = = = = = = = = = = = = = = = = = = = = = = = = = = = = = = = = = = = = = = = = =
\usepackage{graphicx}
\usepackage{lmodern}
\usepackage{url}
\usepackage{subfiles}
\usepackage{pdfpages}
\usepackage{wrapfig}
\usepackage{lipsum}			% Adds Lorem Ipsum placeholder

% = = = = = = = = = = = = = = = = = = = = = = = = = = = = = = = = = = = = = = = = =
% |		CUSTOM PACKAGES
% = = = = = = = = = = = = = = = = = = = = = = = = = = = = = = = = = = = = = = = = =
\usepackage[ihkPotsdam, timeNotesPage, tocSettings, headerAndFooter]{../packages/teltowPack}

% \setThemeColor[003366]
\isExamInSummer[true]
\setAuthor[Mustermann, Max]
\setSpecialization[Fachinformatiker für Anwendungsentwicklung]
\setLoginDetails[ID-24z5n2fxasdf]
\setProjectTitle[Implementierung einer Desktopanwendung zur Verarbeitung und Verwaltung von Übertragungswerten]
\setCompanyName[Bug Guardians GmbH \& Co. KG]
\setSupervisorName[Hanno Nym]
\setSupervisorEmail[nym.hanno@bug.guardians.de]
\setSupervisorTelefon[03328 35070]
\setCompanyLogo[../assets/img/logo-placeholder.png]

% \setHeaderLogoPosition[2.8]
% \setHeaderLogoSize[2.75cm]

% = = = = = = = = = = = = = = = = = = = = = = = = = = = = = = = = = = = = = = = = =
% |		BEGIN DOCUMENT
% = = = = = = = = = = = = = = = = = = = = = = = = = = = = = = = = = = = = = = = = =
\begin{document}
\makeIHKTitlePage%

% \subfile{../pages/test/test.tex} compile all .tex subpages with PowerShell Script: /helper/compile-all.ps1
% OR
\makeTimeNotesPage{
    \entry{01.02.2069}{Projekt Start -- Beginn Plannungsphase -- Recherchen}{2}
    \entry{01.02.2069}{IST-Analyse durchführen (Fachgespräche, etc.)}{3}
    \entry{01.02.2069}{Wirtschaftlichkeitsanalyse und Nutzwertanalyse durchführen}{3}
    \entry{02.02.2069}{Beginn Plannungsphase -- Zielplattform festlegen}{1}
    \entry{02.02.2069}{Benutzeroberfläche Entwerfen}{2}
    \entry{03.02.2069}{Datenbank Diagramm erstellen}{2}
    \entry{03.02.2069}{Plannung der Geschäfftslogik}{2}
    \entry{04.02.2069}{Beginn Implementierungsphase -- Anlegen der Datenbank}{1 $\frac{1}{2}$}
    \entry{05.02.2069}{Implementierung der Geschäfftslogik}{$1,5$}
    \entry{05.02.2069}{... usw.}{2}
    \entry{05.02.2069}{... usw.}{3}
    \entry{06.02.2069}{... usw.}{4}
    \entry{07.02.2069}{... usw.}{6}
    \entry{07.02.2069}{... usw.}{7}
    \entry{09.02.2069}{... usw.}{8}
    \entry{10.02.2069}{... usw.}{7}
    \entry{11.02.2069}{... usw.}{4}
    \entry{12.02.2069}{... usw.}{3}
    \entry{13.02.2069}{... usw.}{2}
    \entry{14.02.2069}{Erstellung Projektdokumentation}{8}
    \entry{14.02.2069}{Erstellung Kundendokumentation}{0,5}
    \entry{10.02.2069}{... usw.}{7}
    \entry{10.02.2069}{... usw.}{6}
    \entry{11.02.2069}{... usw.}{5}
    \entry{11.02.2069}{... usw.}{4}
    \entry{12.02.2069}{... usw.}{3}
    \entry{13.02.2069}{... usw.}{2}
}
\tableofcontents
\newpage
\listoffigures
\addcontentsline{toc}{section}{Abbildungsverzeichnis} % add page to toc
\clearpage
% \listoftables
% \addcontentsline{toc}{section}{Tabellenverzeichnis} % add page to toc
% \clearpage

\section{Einleitung}
    \subsection{Projektbeschreibung}
    \begin{figure}[ht]
        \centering
        \includegraphics[width=.5\textwidth]{example-image}
        \caption[Beispiel Bild]{Beispiel Bild\footnotemark.}
    \end{figure}
    \footnotetext{Beispiel Bild. verfügbar unter \url{www.nowhere.de}, aufgerufen am: garnicht.2024 um 00:00Uhr}
    \lipsum\
    \subsection{Projektziel}
    \lipsum[1-3]
    \begin{wrapfigure}{l}{.3\textwidth}
        \centering
        \includegraphics[width=.3\textwidth]{example-image-a}
        \caption[Klickibunti Mockup]{Klickibunti Mockup\footnotemark.}
    \end{wrapfigure}
    \lipsum[3-6]
    \footnotetext{Klickibunti mockups. verfügbar unter \url{www.nowhere.de}, aufgerufen am: 100 v. Chr. um 07:00Uhr}
    \subsection{Projektumfeld}
    \lipsum\
    \subsection{Projektschnittstellen}
    \lipsum\
\section{Projektplannung}
    \subsection{Projektphasen}
    \lipsum\
    \subsection{Ressourcenplannung}
    \lipsum\
\section{Anlysephase}
    \subsection{Anwendungsfälle}
    \lipsum\
\section{Entwurfsphase}
    \subsection{Zielplattform}
    \lipsum\
    \subsection{Architekturdesign}
    \lipsum\
    \subsection{Entwurf der Benutzeroberfläche}
    \lipsum\
    \subsection{Datenmodell}
    \lipsum\
    \subsection{Geschäftslogik}
    \lipsum\
\section{Implementierungsphase}
\newpage
\section{Dokumentation}
\newpage
\section{Fazit}

\newpage
\begin{alphasection}
\section{Anhang}
\end{alphasection}

%
%       Stats
%
% \newpage
% {\Large Statistics \par}
% \vspace{.5cm}
% \verb|Total Reports:|\arabic{reportID}

\end{document}
