\documentclass[12pt, a4paper]{article}
\usepackage[a4paper, left=2.5cm, right=2cm, top=2cm, bottom=3cm, heightrounded, includefoot]{geometry}

% = = = = = = = = = = = = = = = = = = = = = = = = = = = = = = = = = = = = = = = = =
% |		PACKAGES
% = = = = = = = = = = = = = = = = = = = = = = = = = = = = = = = = = = = = = = = = =
\usepackage{graphicx}
\usepackage{lmodern}
\usepackage{url}
\usepackage{subfiles}       % enables including TeX files
\usepackage{pdfpages}       % enables including pdf files
\usepackage{wrapfig}        % inline images
\usepackage{lipsum}			% Adds Lorem Ipsum placeholder

% = = = = = = = = = = = = = = = = = = = = = = = = = = = = = = = = = = = = = = = = =
% |		CUSTOM PACKAGES
% = = = = = = = = = = = = = = = = = = = = = = = = = = = = = = = = = = = = = = = = =
\usepackage[ihkPotsdam, timeNotesPage, tocSettings, headerAndFooter]{../packages/teltowPack}

% \setThemeColor[003366]
\isExamInSummer[true]
\setAuthor[Mustermann, Max]
\setSpecialization[Fachinformatiker für Anwendungsentwicklung]
\setLoginDetails[ID-24z5n2fxasdf]
\setProjectTitle[Implementierung einer Desktopanwendung zur Verarbeitung und Verwaltung von Übertragungswerten]
\setCompanyName[Bug Guardians GmbH \& Co. KG]
\setSupervisorName[Hanno Nym]
\setSupervisorEmail[nym.hanno@bug.guardians.de]
\setSupervisorTelefon[03328 35070]
\setCompanyLogo[../assets/img/logo-placeholder.png]

% \setHeaderLogoPosition[2.8]
% \setHeaderLogoSize[2.75cm]

% = = = = = = = = = = = = = = = = = = = = = = = = = = = = = = = = = = = = = = = = =
% |		BEGIN DOCUMENT
% = = = = = = = = = = = = = = = = = = = = = = = = = = = = = = = = = = = = = = = = =
\begin{document}
\makeIHKTitlePage%

\includepdf[pages=-]{../assets/Projektantrag.pdf}

% \subfile{../pages/test/test.tex} compile all .tex subpages with PowerShell Script: /helper/compile-all.ps1
% OR
\makeTimeNotesPage{
    \entry{01.02.2069}{Projekt Start -- Beginn Plannungsphase -- Recherchen}{2}
    \entry{01.02.2069}{IST-Analyse durchführen (Fachgespräche, etc.)}{3}
    \entry{01.02.2069}{Wirtschaftlichkeitsanalyse und Nutzwertanalyse durchführen}{3}
    \entry{02.02.2069}{Beginn Plannungsphase -- Zielplattform festlegen}{1}
    \entry{02.02.2069}{Benutzeroberfläche Entwerfen}{2}
    \entry{03.02.2069}{Datenbank Diagramm erstellen}{2}
    \entry{03.02.2069}{Plannung der Geschäfftslogik}{2}
    \entry{04.02.2069}{Beginn Implementierungsphase -- Anlegen der Datenbank}{1 $\frac{1}{2}$}
    \entry{05.02.2069}{Implementierung der Geschäfftslogik}{$1,5$}
    \entry{05.02.2069}{... usw.}{2}
    \entry{05.02.2069}{... usw.}{3}
    \entry{06.02.2069}{... usw.}{4}
    \entry{07.02.2069}{... usw.}{6}
    \entry{07.02.2069}{... usw.}{7}
    \entry{09.02.2069}{... usw.}{8}
    \entry{10.02.2069}{... usw.}{7}
    \entry{11.02.2069}{... usw.}{4}
    \entry{12.02.2069}{... usw.}{3}
    \entry{13.02.2069}{... usw.}{2}
    \entry{14.02.2069}{Erstellung Projektdokumentation}{8}
    \entry{14.02.2069}{Erstellung Kundendokumentation}{0,5}
    \entry{10.02.2069}{... usw.}{7}
    \entry{10.02.2069}{... usw.}{6}
    \entry{11.02.2069}{... usw.}{5}
    \entry{11.02.2069}{... usw.}{4}
    \entry{12.02.2069}{... usw.}{3}
    \entry{13.02.2069}{... usw.}{2}
}
\tableofcontents

\newpage
\section*{Abkürzungsgsverzeichnis}
\addcontentsline{toc}{section}{Abkürzungsgsverzeichnis}
\begin{tabularx}{\linewidth}{lX}
    DB & Database, zu Deutsch Datenbank \\
    SQL & \textbf{S}tructured \textbf{Q}uery \textbf{L}anguage \\
    IDE & \textbf{I}ntegrated \textbf{D}evelopment \textbf{E}nvironment, Entwicklungsumgebung \\
\end{tabularx}

\newpage
\listoffigures
\addcontentsline{toc}{section}{Abbildungsverzeichnis} % add page to toc

\newpage
% \listoftables
% \addcontentsline{toc}{section}{Tabellenverzeichnis} % add page to toc
% \clearpage

\section{Einleitung}
    \subsection{Projektbeschreibung}
    \begin{figure}[ht]
        \centering
        \includegraphics[width=.5\textwidth]{example-image}
        \caption[Beispiel Bild]{Beispiel Bild\footnotemark.}
    \end{figure}
    \footnotetext{Beispiel Bild. verfügbar unter \url{www.nowhere.de}, aufgerufen am: garnicht.2024 um 00:00Uhr}
    \lipsum[1]
    \subsection{Projektziel}
    \lipsum[1-2]
    \begin{wrapfigure}{l}{.3\textwidth}
        \centering
        \includegraphics[width=.3\textwidth]{example-image-a}
        \caption[Klickibunti Mockup]{Klickibunti Mockup\footnotemark.}
    \end{wrapfigure}
    \lipsum[2-3]
    \footnotetext{Klickibunti mockups. verfügbar unter \url{www.nowhere.de}, aufgerufen am: 100 v. Chr. um 07:00Uhr}
    \subsection{Projektumfeld}
    \lipsum[1]

\newpage
\section{Projektplannung}
    \subsection{Lorem Ipsum}
    \lipsum[1]
    \begin{table}[!htb]
        \centering
        \caption{Projektphasen}
        \begin{tabular}{|lr|}
            \hline
            \textbf{Projektphasen} & \textbf{Geplante Zeit} \\
            \hline
            Phase1 & 42h \\
            Phase2 & 42h \\
            Phase3 & 42h \\
            Phase4 & 42h \\
            Phase5 & 42h \\
            Phase6 & 42h \\
            \hline
            Gesamt & 42h \\
            \hline
        \end{tabular}
    \end{table}

\section{Anlysephase}
    \subsection{Lorem Ipsum}
    \lipsum[1-2]
    \subsection{\glqq{}Make or Buy\grqq{}- Entscheidung}
    \lipsum[1]
    \subsection{Projektkosten}
    \begin{table}[h]
        \centering
        \caption{Projektkosten}
        \begin{tabularx}{\linewidth}{|lXrrrr|}
            \hline
            \textbf{Vorgang} & \textbf{Mitarbeiter} & \textbf{Zeit} & \textbf{Personalkosten} & \textbf{Ressourcen} & \textbf{Gesamt}\\
            \hline
            Plannung & 1x Auszubildender & 42h & 42,00€ & 2,00€ & 44,00€ \\
            Entwicklung & 1x Auszubildender & 42h & 42,00€ & 2,00€ & 44,00€ \\
            Code Review & 1x Mitarbeiter & 42h & 1,00€ & 2,00€ & 3,00€ \\
            Abnahme & 1x Mitarbeiter & 42h & 1,00€ & 2,00€ & 3,00€ \\
            \hline
            & & & & & 2920,00€ \\
            \hline
        \end{tabularx}
    \end{table}

\section{Entwurfsphase}
    \subsection{Lorem Ipsum}
    \lipsum[2-3]

\section{Implementierungsphase}
    \subsection{Lorem Ipsum}
    \lipsum[3-4]

\section{Dokumentation}
    \subsection{Entwicklerdokumentation}
    Im \fullref{anhang:Anhang1} ist die Dokumentation für Entwickler zu finden.

\section{Fazit}

\clearpage
\appendix
\section{Anhang}

    \subsection{Anhang 1}\label{anhang:Anhang1}
    \begin{figure}[htb]
        \centering
        \includegraphics[width=.95\textwidth]{example-image}
        \caption{Beispiel Anhang1}
    \end{figure}

    \clearpage
    \subsection{Anhang 2}\label{anhang:Anhang2}
    \begin{figure}[htb]
        \centering
        \includegraphics[width=.95\textwidth]{example-image}
        \caption{Beispiel Anhang2}
    \end{figure}
    
\end{document}
